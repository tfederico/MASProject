\section{Performance}
\subsection{Improvements}
To establish the performance of the final code, a comparison is made to a baseline code in Table \ref{table:table1}, using the first day data. The baseline code, that is our first version of the solution, buses only travel, pick up passengers and drop off them. In the final version, all the improvements explained in Section \ref{sec:conceptual} are implemented.

\begin{table*}[htbp]
\centering
\begin{tabular}{ |c|c|c|  }
 \hline
  Measurement & Baseline & Final Code \\
 \hline
  Avg travel time & 150 & 72.53 \\
  Buses' expenses & 1039682 & 711251 \\
  Messages sent & 0 & 285  \\
  Final amount waiting & 232 & 255 \\
  Avg travel time remaining & 100 & 78.12 \\
  Final avg travel time & 156 & 75.48 \\
 \hline
\end{tabular}
\label{table:table1}
\caption{Performance results}
\end{table*}

Table \ref{table:table1} shows that there is a significant decrease in the average travelling times and in the expenses. Obviously, the number of messages increase (using the first version, no message was sent). However, the number of people waiting at the end of the simulation slightly increases in the new version.

This result highlights the advantages of an intelligent coordination and competion between buses compared to a system of buses that does not exploit these concepts.

\subsection{Perfomance on test data}

For this project, we have three different data set, corresponding to three different days. Day 1 is used to improve our solution, while Day 4 and 5 are used to test our solution. In this section, we show the results of our solution in the three different data sets.

The final value of the different metrics for the dataset of Day 1 are showed in the right column of Table \ref{table:table1}, whereas the plots for the expenses, the messages, the number of people waiting and the travelling time are shown in Figure \ref{fig:expense}, \ref{fig:messages}, \ref{fig:pass_waiting} and \ref{fig:avg_tt} respectively.

\begin{figure}[htbp]
\centering
\begin{minipage}{.48\textwidth}
  \includegraphics[width=\textwidth]{src/expenses.png}
  \caption{\label{fig:expense}Expenses of the buses in Day 1.}
\end{minipage}%
\begin{minipage}{.48\textwidth}
  \includegraphics[width=\textwidth]{src/nr_messages.png}
  \caption{\label{fig:messages}Number of exchanged messages in Day 1.}
\end{minipage}
\end{figure}

\begin{figure}[htbp]
\centering
\begin{minipage}{.48\textwidth}
  \includegraphics[width=\linewidth]{src/nr_pass_waiting.png}
  \caption{\label{fig:pass_waiting}Final number of passengers waiting in Day 1.}
\end{minipage}%
\begin{minipage}{.48\textwidth}
  \includegraphics[width=\linewidth]{src/avg_tt.png}
  \caption{\label{fig:avg_tt}Average travelling time in Day 1.}
\end{minipage}
\end{figure}

Using the dataset of Day 1, we can see that there are two peeks in which the number of people suddenly appear in the map, and in which the fleet have to deal with this situation. It is important to note that during the first peek, our solution creates a number of buses that is probably enough to deal with the second peek (we can see a sudden increase in the bus expenses when there is the first peek, followed by a constant increase of the expenses for the rest of the simulation).

We are now going to focus on the results using the test sets, namely Day 4 and Day 5. In Table \ref{table:table2}, the metrics value on the two test sets are shown.

\begin{table*}[htbp]
\centering
\begin{tabular}{ |c|c|c|  }
 \hline
  Measurement & Test Data set 2 & Test Data set 3 \\
 \hline
  Avg travelling time & 92.36 & 86.72 \\
  Buses' expenses & 2393732 & 2346943.5 \\
  Messages sent & 950 & 992 \\
  Final amount waiting & 2603 & 2513 \\
  Avg travel time remaining & 182.97 & 206.91 \\
  Final avg travel time & 101.42 & 95.97 \\
 \hline
\end{tabular}
\label{table:table2}
\caption{Performance results}
\end{table*}

Our solution experienced a lot more difficulties in managing the situations of the test set. While the average travelling time is slighlty increased, compared to the one on the training set, we can notice a strong increase in the expenses and in the final amount of passengers waiting. 

The plots using Day 4 for the expenses, the messages, the number of people waiting and the travelling time are shown in Figure \ref{fig:expense4}, \ref{fig:messages4}, \ref{fig:pass_waiting4} and \ref{fig:avg_tt4} respectively.

\begin{figure}[htbp]
\centering
\begin{minipage}{.48\textwidth}
  \includegraphics[width=\textwidth]{src/expenses4.png}
  \caption{\label{fig:expense4}Expenses of the buses in Day 4.}
\end{minipage}%
\begin{minipage}{.48\textwidth}
  \includegraphics[width=\textwidth]{src/nr_messages4.png}
  \caption{\label{fig:messages4}Number of exchanged messages in Day 4.}
\end{minipage}
\end{figure}

\begin{figure}[htbp]
\centering
\begin{minipage}{.48\textwidth}
  \includegraphics[width=\linewidth]{src/nr_pass_waiting4.png}
  \caption{\label{fig:pass_waiting4}Final number of passengers waiting in Day 4.}
\end{minipage}%
\begin{minipage}{.48\textwidth}
  \includegraphics[width=\linewidth]{src/avg_tt4.png}
  \caption{\label{fig:avg_tt4}Average travelling time in Day 4.}
\end{minipage}
\end{figure}

The plots using Day 5 for the expenses, the messages, the number of people waiting and the travelling time are shown in Figure \ref{fig:expense5}, \ref{fig:messages5}, \ref{fig:pass_waiting5} and \ref{fig:avg_tt5} respectively.

\begin{figure}[htbp]
\centering
\begin{minipage}{.48\textwidth}
  \includegraphics[width=\textwidth]{src/expenses5.png}
  \caption{\label{fig:expense5}Expenses of the buses in Day 5.}
\end{minipage}%
\begin{minipage}{.48\textwidth}
  \includegraphics[width=\textwidth]{src/nr_messages5.png}
  \caption{\label{fig:messages5}Number of exchanged messages in Day 5.}
\end{minipage}
\end{figure}

\begin{figure}[htbp]
\centering
\begin{minipage}{.48\textwidth}
  \includegraphics[width=\linewidth]{src/nr_pass_waiting5.png}
  \caption{\label{fig:pass_waiting5}Final number of passengers waiting in Day 5.}
\end{minipage}%
\begin{minipage}{.48\textwidth}
  \includegraphics[width=\linewidth]{src/avg_tt5.png}
  \caption{\label{fig:avg_tt5}Average travelling time in Day 5.}
\end{minipage}
\end{figure}
