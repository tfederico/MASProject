\section{Discussion and conclusion}
In this project, we implemented a multi-agents system composed out of a set of buses. The buses interact with each other in order to transport passengers in the city of Amsterdam. 

The behaviour of buses is managed through its knowledge about the environment (belief), goals (intentions) and responsibility within the fleet (role). Each bus travels in a specific area of the map, that is associated with a set of bus stops. Furthemore, within the same area buses can move along two different directions. This enables intelligent and efficient management of passengers. 

When the number of waiting passengers increase, new buses are created. The local coordinators negotiate the distribution of new buses between areas. When one coordinator faces a shortage of buses, whereas other areas have remnant buses, they cooporate. Local coordinators from different areas support each other by exchanging buses between areas.

This solution improved travel time, expenses and the number of people waiting. It is challenging for transportation companies to reduce expenses, without making worse the travel time and the number of waiting passengers. An intelligent multi-agent bus mechanism could possibly lower both. 

Our solution is based on a form of transportation, which entails that buses travel along a fixed route. We suspect that a solution in which travelling is based on the position and destination of the passengers, can lead to improved results. However, such a solution could be difficult to extend and to interpret, whereas a schedule-based solution is particularly easy to improve with new ideas and new behaviours. 

For future works, it would be useful to perform a detailed study of the behaviour of our system in different scenarios. This, in order to understand when and why the system is not working adequately. This study could highlight problems in the conceptual model that we designed, and could help to increase the performance of our solution.
