\section{Conceptual description and strategy}
\label{sec:conceptual}

In this section, we explain the various agent-related concepts that we use, how we design and implement them, and along which strategies. Four types of costs shape our strategies:
\begin{enumerate*}[label=(\roman*)]
\item the amount of money we spend;
\item the amount of messages buses send;
\item the amount of people waiting for a bus;
\item the average amount of time a person spends waiting for a bus.
\end{enumerate*}

Costs arise by traveling and buying more buses. To reduce costs, the buses need to transport travelers efficiently. The more passengers they can transport in the shortest amount of time, the lower the costs are.

\subsection{Transportation and Coordination}

Buses travel according to a specific schedule. These different schedules describe the connections between various bus stops. In the following paragraphs, we focus on how we chose the different schedules and how the buses travel through them.
 
\paragraph{Different fixed schedules}

In order to reduce the problem into smaller sub-problems, we split the map four different \textit{areas}. These areas can be classified under the labels \textit{``West'', ``North'', ``Center'' and ``East''}. Thus, each area defines a set of bus stops, and each bus is assigned to a specific area. Each set has one or more elements (i.e., bus stops) in common with at least another set.
Splitting the bus stops into sets has two advantageous effects. For example, a bus focuses only in its specific area: thus, it has to complete a shorter distance before passing the same stop again, so the waiting time is reduced. Moreover, the time that a traveler spends in the bus will be shorter, which will reduce chances of buses being full and unable to pick up other travelers.

This solution occasionally requires transferring passengers from one area to another. This reduces the advantageous effects. To discuss about transportation of people between bus stops in different schedules, we define the concept of \textit{joint position}. A joint position is a bus stop that is shared between different areas. For example, if the bus stop labeled as 4 is shared between Area 1 and Area 2, we say that \textit{bus stop 4} is a joint position  between Area 1 and  Area 2. Stop 3 (\texttt{Centraal}) is a shared stop for all four areas. As a result, when a bus picks up a passenger that needs to change area, it will consider the nearest joint position with the desired area as the passenger destination: the agent brings the passenger there and it drops it. Then, an agent from the destination area picks up the passenger from the joint position and it brings the passenger to his real destination. In order to avoid useless movements of passenger, we define some conditions to pick up a passenger from a bus stop:

\begin{itemize}
\item the passenger destination is in the bus area, \textbf{or}
\item the passenger destination is not in the bus area \textbf{and} the passenger is not already in a joint position with the destination area.
\end{itemize}

\paragraph{Direction-based pick up}

Within an area, we define two different schedules. The first one is a route through all the bus stops of the area, the second one is the reverse of the first one. Thus, we can have bus traveling in the same area but following two different directions. Consequently, we improve the traveling time by moving passengers along the direction that is faster in order to reach the destination. This adds a new condition to check whether a bus should pick a passenger to the ones listed in the previous paragraph: the passenger has to be picked up if and only if it is faster to move the passenger to the destination (or to the joint position) using the current bus schedule, compared to the reverse one. If this condition does not hold, the passenger is not picked up, and he/she waits for another bus that is traveling with the opposite schedule. This increases transportation efficiency: since buses carry passengers for a shorter amount of time, they are able to carry more passengers. 

The number of buses assigned to areas and schedules is distributed based on the needs of that specific schedule, as we explain later. In this way, schedule with higher need of buses are provided with new buses.

\subsection{Roles}
\label{subsec:roles}

To ensure that the buses cooperate efficiently, we assign them specific roles. These roles differ in their ancillary responsibilities and so a hierarchy of buses was created.

Roles can also define the \textit{behaviour} of a bus. The usage of roles makes it easy to change the behaviour of a specific group of buses.

In addition, the roles define the creation time of a bus. At the beginning of the simulation, there is only one bus, with ID 24. However, as soon as it is able to create new buses (i.e., after few ticks), this bus creates another seven initial buses: three local coordinators and four scouts, as we explain in the next paragraphs.

\paragraph{Global coordinator}

A major responsibility is buying new buses in case of shortage. If all buses would be able to do that, it would happen inefficiently and increase costs unnecessarily. Therefore we assign this responsibility only to the first bus created, the one with ID 24. This bus is on top of the hierarchy, and its role is defined as \textit{global coordinator}. 

The \textit{global coordinator} adds buses based on shortage. Shortage is defined as the difference between the total number of people waiting and the total capacity of the bus fleet. It adds the cheapest bus type that is able to manage the amount of waiting people. Thus, when a shortage is experienced, the global coordinator creates new buses: as we explain later on, the newly created buses are going to be items of auctions in order to be assigned to a specific schedule.

\paragraph{Local coordinators}

One step down in the hierarchy, we have the \textit{local coordinators}. Each of them is responsible for one of the four areas in Amsterdam. Note that the \textit{global coordinator} is also a \textbf{local coordinator}. The \textit{global coordinator} is assigned to the "Centre", whereas the buses with ID 25, 26 and 27 are responsible for ``West'', ``East'' and ``North'' respectively. The glocal coordinator creates all of them at the beginning of the simulation. The local coordinators keep track of their area by listing the buses assigned to their area, their fleet capacity and the number of people waiting. When a new bus is created, all the local coordinators compete between each other to assign the new bus in their area, as described in Section~\ref{subsec:negotiations}.

\paragraph{Scouts}

Less important than the local coordinators are the \textit{scouts}. Note that the global coordinator and the local coordinators are also scouts. They are created at the beginning of the simulation: each schedule has one scout. Their aim is to have a bus that is always moving in the schedule. They do not have specific responsibility.

\paragraph{Simple}

Finally, at the bottom of the hierarchy we have the simple buses. Every bus is also a simple bus. However, the bus created after the beginning of the simulation are \textit{exclusively} simple buses. This means that they do not have specific things to do, but they only have to pick, move and drop passengers.

\subsection{Communication}

As previously mentioned, to enable social interaction between agents with conflicting or agreeable goals, we implemented a form of communication. This communication happens through the exchange of messages. A bus sends a message with a specific goal, and the receiver performs actions based on the content.

\paragraph{Ontology}

To take into account the possibility of messages with different purpose, we define the following ontology for communication: \texttt{"message\_type content"}, where  \texttt{message\_type} is the header of the message, or the performative~\cite{fipa}. This is a string that describes the goal of the message. \texttt{content} contains the effective content of the message. In this way, buses know how to use the content of the message based on the performative. The perfomatives of the messages can be related to: promotions, assigning buses to schedules, requesting/offering/accepting help, requesting a bid, bidding and reallocating a bus. 

The reception of the messages goes as follows. At every tick, a bus checks its incoming messages. To check whether a message is new, its tick is compared to the latest tick in which the agent checked its inbox. This last tick is represented by a locally stored variable. The performative of the message implies how to interpret the content. Based on the content, the bus might changes its intentions. For example, when a local coordinator gets a message of type ``\textit{bid-request}'', the new intention of performing a bid is added.

\subsection{Group decisions}

The division of the route into four areas can lead to certain scenarios that involve unnecessary costs. We take steps to deal with a situation, where one area has a shortage of buses whereas the other area has a remnant. 

Whenever a bus is created and allocated to a new schedule, the coordinator sends a message to the corresponding local coordinator, in order to make it keep track of all buses that are working in that area. On the other hand, local coordinators compare the amount of people waiting in their area with the fleet capacity. Based on this comparison, they decide whether they need more buses. If this is the case, they collaborate with the other coordinators. 

We implemented this collaboration by sending a message to the other local coordinators. The message has performative \texttt{help-request} and the content is a number, which represents the difference between the local fleet capacity and the amount of people waiting in that area. The other coordinators ask to their fleets if there are empty bus which can be transferred (i.e., performative \texttt{empty-check}), and if this is the case the empty buses send a message with performative \texttt{reallocable} and content their ID to the bus that requested help. Upon receiving this offer, the requester evaluates how many buses it needs to compensate the difference between the fleet capacity and the amount of people waiting, adding these buses to its fleet and sending them a message with performative \texttt{reallocated} and with content corresponding to the local coordinator id. On the contrary, the ones that are not reallocated receive a message with performative \texttt{disengage}, which tells them that they do not need to change schedule. The other coordinator removes the bus from its bus fleet when it receives a message with performative \texttt{bye-bye}.

\subsection{Negotiation}
\label{subsec:negotiations}

The local coordinators compete between each others in order to obtain the newly created buses.

When the global coordinator creates a new bus, it also sends a message \texttt{bid-request} to every local coordinator, that starts the auction for the new bus. Every local coordinator sends a message \texttt{bid} to the local coordinator specifying the need of the two schedules in its area. The need for a specific schedule is computed counting the number of passengers in the area that should be picked up by a bus following the schedule (remember that, based on the schedule, the direction of the bus changes and passengers are picked up based on the traveling direction). After receiving the bids from the local coordinators, the global coordinator chooses the schedule for which the need is higher, and it specifies to the newly created bus that it should drive using that schedule, sending it a message of type \texttt{schedule} containing the ID of the schedule.

\subsection{Stopping buses}
\label{sec:stop}

We allow a bus to stop when it matches some conditions. In this way, we can reduce the traveling costs whenever it is possible. In particular, a bus can stop when:

\begin{itemize}
\item the bus is empty \textbf{and}
\item there are no passengers waiting in its schedule; here, we consider the number of passengers waiting in the bus stops composing the schedule of the bus that have to be picked up (considering the direction of the bus).
\end{itemize}
