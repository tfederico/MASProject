\section{Conceptual description and strategy}
\label{sec:conceptual}

In this section, we explain the various agent-related concepts that we use, how we design and implement them, and along which strategies. Four types of costs shape our strategies:
\begin{enumerate*}[label=(\roman*)]
\item the amount of money we spend;
\item the amount of messages buses send;
\item the amount of people waiting for a bus;
\item the average amount of time a person spends waiting for a bus.
\end{enumerate*}

Costs arise by traveling and buying more buses. To reduce costs, the buses need to transport travelers efficiently. The more passengers they can transport in the shortest amount of time, the lower the costs are.

\subsection{Transportation and Coordination}

Buses travel according to a specific schedule. These different schedules describe the connections between various bus stops. In the following paragraphs, we focus on how we chose the different schedules and how the buses travel through them.
 
\paragraph{Different fixed schedules}

To create more easily managed transportation areas, we split the map into four different \textit{areas}. These areas can be classified under the labels \textit{``West'', ``North'', ``Center'' and ``East''}. Each area is defined by a schedule of bus stops, and each bus is assigned to a specific area. Each schedule has at least one bus stop in common with another schedule.

Splitting the bus stops into sets has two advantageous effects. Firstly, buses focus on the specific area they are assigned to. This means that it has to complete a shorter distance before passing the same stop again. Thus the waiting time for travelers is reduced. Moreover, the time that a traveler spends in the bus will be shorter, which will reduce chances of buses being full and unable to pick up waiting travelers.

This solution occasionally requires transferring passengers from one area to another. This reduces the advantageous effects somewhat. For transportation of people between bus stops in different schedules, we define the concept of a \textit{joint position}. A joint position is a bus stop that is shared between different areas. Bus stop 4 for example is in the schedule of area \textit{``West''} and in the schedule of area \textit{``Centre''}. So we say that \textit{bus stop 4} is a joint position between those areas. Stop 3 (\texttt{Centraal}) is a joint position between all four areas. The joint positions make it possible to transfer passengers from one area to another. When a bus picks up a passenger with a destination in a different area, it will check for the nearest joint position with that other area. After the passenger is transported and dropped off at the joint stop, a bus from the destination area picks up the passenger from the joint position. This second bus can bring the passenger to the final destination. In order to transfer passengers efficiently, we define some conditions to pick up a passenger from a bus stop:

\begin{itemize}
\item the passenger destination is in the bus area, \textbf{or}
\item the passenger destination is not in the bus area \textbf{and} the passenger is not already in a joint position with the destination area.
\end{itemize}

\paragraph{Direction-based pick up}

Within an area, we define two different schedules. The first one is a route through all the bus stops of the area, the second one is the reverse of the first one. This bidirectional transportation helps to lower travel time. Passengers are picked up by buses that will reach their destination faster, due to the direction they are traveling in. To profit from bidirectional transportation, we add a third condition for pick-up to the list in the previous paragraph. If the bus has to travel fewer stops to reach the destination of a passenger, than a bus traveling in the opposite direction, a passenger is picked up. The destination can be the final destination of the passenger, or the joint position. If this condition does not hold, the passenger is not picked up. As soon as a bus arrives that travels in the opposite direction, the passenger will be picked up. This increases transportation efficiency; buses carry passengers for a shorter amount of time and they are able to carry more passengers. 

\subsection{Roles}
\label{subsec:roles}

To ensure that the buses cooperate efficiently, they are assigned roles. These roles differ in their ancillary responsibilities, which resulted in a bus hierarchy.

Besides responsability, roles can define the \textit{behaviour} of a bus. The usage of roles makes it easy to change the behaviour of a specific group of buses.

In addition, the roles define the creation time of a bus. At the beginning of the simulation, there is only one bus, with ID 24. However, as soon as it is able to create new buses (i.e., after few ticks), this bus creates another seven buses. These are three local coordinators and four scouts, their roles will be explained in this section.

\paragraph{Global coordinator}

A major responsibility is buying new buses in case of shortage. If all buses would be able to do that, it would happen inefficiently and increase costs unnecessarily. Therefore we assign this responsibility only to the first bus created, the one with ID 24. This bus is on top of the hierarchy, and its role is defined as \textit{global coordinator}. 

The \textit{global coordinator} adds buses based on shortage. Shortage is defined as the difference between the total number of people waiting and the total capacity of the bus fleet. It adds the cheapest bus type that is able to manage the amount of waiting people. Thus, when a shortage is experienced, the global coordinator creates new buses: as we explain later on, the newly created buses are going to be items of auctions in order to be assigned to a specific schedule.

\paragraph{Local coordinators}

One step down in the hierarchy, we have the \textit{local coordinators}. Each of them is responsible for one of the four areas in Amsterdam. Note that the \textit{global coordinator} is also a \textbf{local coordinator}. The \textit{global coordinator} is assigned to the ``Centre'', whereas the buses with ID 25, 26 and 27 are responsible for ``West'', ``East'' and ``North'' respectively. The glocal coordinator creates all of them at the beginning of the simulation. The local coordinators keep track of their area by listing the buses assigned to their area, their fleet capacity and the number of people waiting. When a new bus is created, all the local coordinators compete between each other to assign the new bus in their area, as described in Section~\ref{subsec:negotiations}.

\paragraph{Scouts}

Less important than the local coordinators are the \textit{scouts}. Note that the global coordinator and the local coordinators are also scouts. They are created at the beginning of the simulation: each schedule has one scout. Their aim is to have a bus that is always moving in the schedule. They do not have specific responsibility.

\paragraph{Simple}

Finally, at the bottom of the hierarchy we have the simple buses. Every bus is also a simple bus. However, the bus created after the beginning of the simulation are \textit{exclusively} simple buses. This means that they do not have specific things to do, but they only have to pick, move and drop passengers.

\subsection{Communication}

As previously mentioned, to enable coorporation and competition, we implemented a form of communication. This happens through the exchange of messages. A bus sends a message with a specific goal, and the receiver performs actions based on the content.

\paragraph{Ontology}

To take into account the possibility of messages with different purposes, we define the following ontology for communication: \texttt{"message\_type content"}, where  \texttt{message\_type} is the header of the message, or the performative~\cite{fipa}. This is a string that describes the goal of the message. \texttt{content} contains the effective content of the message. In this way, buses know how to use the content of the message based on the performative. The perfomatives of the messages can be related to: promotions, assigning buses to schedules, requesting/offering/accepting help, requesting a bid, bidding and reallocating a bus. 

The reception of the messages goes as follows. At every tick, a bus checks its incoming messages. To check whether a message is new, its tick is compared to the latest tick in which the agent checked its inbox. This last tick is represented by a locally stored variable. The performative of the message implies how to interpret the content. Based on the content, the bus might changes its intentions. For example, when a local coordinator gets a message of type ``\textit{bid-request}'', the new intention of performing a bid is added.

\subsection{Group decisions}

The division of the route into four areas can lead to certain scenarios that involve unnecessary costs. We take steps to deal with a situation, where one area has a shortage of buses whereas the other area has a remnant. 

Local coordinators watch out for shortages, by comparing their fleet capacity to the amount of people waiting. Whenever a bus is created and allocated to a schedule, the corresponding local coordinator receives a message. With this information, the local coordinators keep track of all the buses that are assigned to their area. This is how they can keep track of their fleet capacity. Based on the comparison of fleet capcity and amount of people waiting in their area, they decide whether they need more buses. If this is the case, they collaborate with the other coordinators. 

The coordinators collaborates by distributing buses between them, based on who needs them the most. This collaboration happens through exchanging messages. When a local coordinator faces a shortage, it sends a message to the other coordinators. The message has performative \texttt{help-request} and the content is a number. This number represents the difference between the local fleet capacity, and the amount of people waiting in their area. The other coordinators ask to the buses in their fleet if they are empty (i.e., performative \texttt{empty-check}). If a bus is empty, it can be transferred to the area that needs support. The empty bus sends a message with performative \texttt{reallocable} and their ID as content, to the coordinator that requested help. Upon receiving this offer, the requester evaluates how many buses it needs to resolve the shortage in the area. If the offer is welcome, the coordinator adds the bus to its fleet. It does so by sending a message with performative \texttt{reallocated} and its coordinator ID to the offered bus. The offered bus changes coordinator and will travel according to the different schedule. The old coordinator will remove this bus from its fleet, after it receives a message with performative \texttt{bye-bye}. Buses whose offer is rejected, receive a message with performative \texttt{disengage}. That tells them that they do not need to change schedule. 

\subsection{Negotiation}
\label{subsec:negotiations}

The local coordinators compete between each others in order to obtain the newly created buses.

When the global coordinator creates a new bus, it also sends a message \texttt{bid-request} to every local coordinator, that starts the auction for the new bus. Every local coordinator sends a message \texttt{bid} to the global coordinator specifying the need for the two schedules in its area. The need for a specific schedule is computed counting the number of passengers in the area that should be picked up by a bus following the schedule (remember that, based on the schedule, the direction of the bus changes and passengers are picked up based on the traveling direction). After receiving the bids from the local coordinators, the global coordinator chooses the schedule for which the need is higher. It specifies to the newly created bus that it should drive according to that schedule, sending it a message of type \texttt{schedule} containing the ID of the schedule.

\subsection{Stopping buses}
\label{sec:stop}

We allow a bus to stop when it matches some conditions. In this way, we can reduce the traveling costs whenever it is possible. In particular, a bus can stop when:

\begin{itemize}
\item the bus is empty \textbf{and}
\item there are no passengers waiting in its schedule; here, we consider the number of passengers waiting in the bus stops composing the schedule of the bus that have to be picked up (considering the direction of the bus).
\end{itemize}
