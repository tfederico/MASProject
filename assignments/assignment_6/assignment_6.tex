\documentclass[a4paper]{article}

\usepackage[english]{babel}
\usepackage[utf8]{inputenc}
\usepackage{amsmath}
\usepackage{graphicx}
\usepackage{listings}
\usepackage{algorithmic}
\usepackage[ruled]{algorithm2e} % For algorithms
\usepackage[colorinlistoftodos]{todonotes}
\usepackage[inline]{enumitem}

\usepackage[a4paper,bindingoffset=0.2in,%
            left=1in,right=1in,top=1in,bottom=1in,%
            footskip=.25in]{geometry}

\title{Multi-Agent systems - Assignment 6 (Competition)}

\author{Emma Machielse, Federico Tavella, Alessandro Tezza}

\date{\today}

\begin{document}
\maketitle

\section{Introduction}
The current version has a competition strategy implemented, where buses bid for the allocation of newly added buses to their area.

\section{Bidding} 

Our transportation system is divided into four areas that each have a bus assigned as a coordinator for this area. These four coordinators ("local coordinators"), have the responsibility to check whether the amount of passengers waiting in their area does not rise above a certain threshold. One of these coordinators, the "global coordinator", also performs this check for the four areas altogether. This global coordinator creates new buses based on this global check if necessary. 
\newline
This is when the bidding happens. When a global coordinator creates a new bus, it sends a message to the other local coordinators to bid. They check the difference between the amount of people waiting in their area and the local fleet capacity, and send this difference as a bidding message to the global coordinator. The global coordinator checks which bid is the highest, i.e. which local coordinator is in most need of a new bus. It assigns the new bus to the area of the local coordinator with the highest bid. 
\newline
The assigning of new buses to the area happens through messaging. The global coordinator sends a message with the coordinator number to the local coordinator. Buses that do not have a role of local or global coordinator, are assigned the role of scout or simple after their creation. When a simple or scout bus is created, after it is assigned a role, it checks it inbox for a schedule. It waits at busstop [3] for assignment to a coordinator. Based on its coordinator number it can retreive its schedule.

\section{Other improvements}
\begin{itemize}
\item When no passengers are waiting, all buses stop travelling, to reduce travelling costs
\item The code is refactored, the agents.nls file is split into 5 files. The agents.nls code describes the core BSI framework, linked to action execution. Then there is an init file, that states how the first buses are created and initialised. The execute-intentions file describes the action functions needed for each intention. The messages file incorporates how our message system works; how to extract messages that are received. Finally we have a utils file that contains functions that are used in other function to extract information.  
\end{itemize}

\end{document}
