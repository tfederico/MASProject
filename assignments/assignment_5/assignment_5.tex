\documentclass[a4paper]{article}

\usepackage[english]{babel}
\usepackage[utf8]{inputenc}
\usepackage{amsmath}
\usepackage{graphicx}
\usepackage{listings}
\usepackage{algorithmic}
\usepackage[ruled]{algorithm2e} % For algorithms
\usepackage[colorinlistoftodos]{todonotes}
\usepackage[inline]{enumitem}

\usepackage[a4paper,bindingoffset=0.2in,%
            left=1in,right=1in,top=1in,bottom=1in,%
            footskip=.25in]{geometry}

\title{Multi-Agent systems - Assignment 5 (Group decisions and coalitions)}

\author{Emma Machielse, Federico Tavella, Alessandro Tezza}

\date{\today}

\begin{document}
\maketitle

\section{Introduction}

For this week assignment we were required to implement a form of group decision and/or coordination. Consequently, all buses must be able to communicate and change their behaviour based on these decisions.

\section{Communication ontology}

During the previous weeks, we defined a really simple form of communication in order to promote (or demote) the role of a bus using a series of labels defined by our own. However, this form o communication is not sufficient for this assignment, since it does not take in account the possibility of messages with different purpose. Thus, we define a basic ontology for communication: \texttt{"message\_type content"}. \texttt{message\_type} is the header of the message, indicating which kind of communication is contained in the string. On the other hand, \texttt{content} indicates the effective content of the message. For example, if we want to promote a bus to the role \textit{scout}, the message would be in the form \texttt{"promotion scout"}.

\section{Group decision \& Coordination}

We implement the request of a new bus as a form of coordination. First of all, we add a new role (i.e., \textit{local coordinator}) for the buses, which implies that a specific bus has the role of a coordinator for the schedule to which it belongs. It keep tracks of all the buses that move in that specific area. In addition, the local coordinator compares the amount of people waiting in its area with the capability of the buses, deciding whether or not a new bus is needed (i.e., the capability is smaller than the amount of people waiting). If this is the case, it ask for help to the local coordinators of different areas. Consequently, the other local coordinators check if in their area there are buses that can fit the request (i.e., that are not coordinators) and send the results to the requester. Once the original bus receives a message containing the \texttt{ID} of a bus that can help it, it sends a message to that bus and to its coordinator to make it change schedule of the available bus. 

The decision of whether to create or not a new bus is still centralized to a specific bus (i.e., the one with the \textit{coordinator role}) based on the global amount of people and capability. However, combining this form of communication and coordination we will be able to implement also a form of group decision in which the \textit{local coordinators} are the one deciding if a new bus in their area is needed by asking to the other responsible buses if they can lend one or more vehicles. In our opinion, this is going to reduce significantly the costs associated to the creation of new buses.

\end{document}
