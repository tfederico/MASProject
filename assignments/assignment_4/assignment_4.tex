\documentclass[a4paper]{article}

\usepackage[english]{babel}
\usepackage[utf8]{inputenc}
\usepackage{amsmath}
\usepackage{graphicx}
\usepackage{listings}
\usepackage{algorithmic}
\usepackage[ruled]{algorithm2e} % For algorithms
\usepackage[colorinlistoftodos]{todonotes}
\usepackage[inline]{enumitem}

\usepackage[a4paper,bindingoffset=0.2in,%
            left=1in,right=1in,top=1in,bottom=1in,%
            footskip=.25in]{geometry}

\title{Multi-Agent systems - Assignment 4 (Coordination)}

\author{Emma Machielse, Federico Tavella, Alessandro Tezza}

\date{\today}

\begin{document}
\maketitle

\section{Introduction}
For this week assignment we were required to implement coordination. This entails that not all buses go to the same bus stop. We tried to achieve this by refining the schedule and by assigning different roles to buses. 

\section{Schedule refinement}

Using one fixed schedule implies that all buses follow the same route and pass by the same bus stops. We improve the fixed schedule by making it bidirectional. Buses follow the same route, but in opposite directions. They only pick up passengers that have a destination closer in their direction than in the other. This increases transportation efficiency, since buses carry passengers for a shorter amount of time and are thus able to carry more passengers. 

Also, we switch from one schedule to multiple smaller schedules. We divide the area into four schedules, that can be classified under the labels "West", "North", "Center" and "East". Because the schedules are shorter, chances of people having to wait because a bus is full are smaller, thus increasing transportation efficiency. To enable transportation of people between bus stops in different schedules, we created a transfer station. Stop 3 (\texttt{Centraal}), is a shared stop for all four schedules. If a bus picks up a passenger with a destination that is not in its schedule, it drops the passenger off at the shared stop. At the shared stop, buses only pick up passengers with a destination that is in their schedule. 

\section{Roles}

We defined different roles for the buses that entail different movement strategies. This is a form of coordination, since it results in the buses moving to different bus stops. The initial bus with \texttt{ID = 24} has the role of coordinator, making this bus responsible for assigning roles and adding buses. Adding busses is based on the amount of people waiting and the total capacity of the bus fleet. The roles that can be assigned to the other buses through communication are "scout" and "simple". A scout is always moving and it is not allowed to stop and wait. Simple buses adhere to a rule of having to stop moving whenever they are empty and they reach an empty bus stop. Outside of peak hours, the coordinator allows for simple empty buses to avoid from pointlessly adding to transportation costs.  

\end{document}
