\documentclass[a4paper]{article}

\usepackage[english]{babel}
\usepackage[utf8]{inputenc}
\usepackage{amsmath}
\usepackage{graphicx}
\usepackage{listings}
\usepackage{algorithmic}
\usepackage[ruled]{algorithm2e} % For algorithms
\usepackage[colorinlistoftodos]{todonotes}
\usepackage[inline]{enumitem}

\usepackage[a4paper,bindingoffset=0.2in,%
            left=1in,right=1in,top=1in,bottom=1in,%
            footskip=.25in]{geometry}

\title{Multi-Agent systems - Assignment 4 (Coordination)}

\author{Emma Machielse, Federico Tavella, Alessandro Tezza}

\date{\today}

\begin{document}

\section{Intro}
For this weeks assignment we were required to implement coordination. According to the assignment this entails that not all buses go to the same busstop. We tried to achieve this by refining the schedule, and by assigning different roles to buses. 

\section{Schedule refinement}
Using one fixed schedule implies that all buses follow the same route and pass by the same busstops. We improved the fixed schedule by making it bidirectional. Buses follow the same route, but in opposite directions. They only pick up passengers that have a destination closer in their direction than in the other. This increases transportation efficiency, since buses carry passengers for a shorter amount of time and are thus able to carry more passengers. 
\newline
Also, we went from one schedule to multiple smaller schedules. We divided the area into four schedules, that could be classified under the headers "West", "North", "Center" and "East". Because the schedules are shorter, chances of people having to wait because a bus is full are smaller, thus increasing transportation efficiency. To enable transportation of people between busstops in different schedules, we created a transfer station. Stop 3 ("Centraal"), is a shared stop for all four schedules. If a bus picks up a passenger with a destination that is not in its schedule, it will drop the passenger off at the shared stop. At the shared stop, buses will only pick up passengers with a destination that is in their schedule. 

\section{Roles}
We defined different roles to the buses that entail different movement strategies. This is a form of coordination, since it results in the buses moving to different bus stops. The initial bus 24 has the role of coordinator, making this bus responsible for assigning roles and adding buses. Adding busses is based on the amount of people waiting and the total capacity of the bus fleet. The roles that can be assigned to the other buses throug communication are that of "scout" and "simple". A scout is always moving and not allowed to stop and wait. Simple buses adhere to a rule of having to stop moving whenever they are empty and they reach an empty bus stop. Outside of peak hours, the coordinator allows for simple buses to avoid empty buses from pointlessly adding to transportation costs.  

\end{document}
